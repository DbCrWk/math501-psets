\documentclass[letterpaper]{article}

\usepackage{amsmath}
\usepackage{amssymb}
\usepackage{hyperref}
\usepackage{geometry}
\usepackage{nth}

% Comment the following lines to use the default Computer Modern font
% instead of the Palatino font provided by the mathpazo package.
% Remove the 'osf' bit if you don't like the old style figures.
\usepackage[T1]{fontenc}
\usepackage[sc,osf]{mathpazo}

\newcommand*{\QED}{\hfill\ensuremath{\square}}%

\makeatletter
\renewcommand{\@seccntformat}[1]{%
  \ifcsname prefix@#1\endcsname
    \csname prefix@#1\endcsname
  \else
    \csname the#1\endcsname\quad
  \fi}
% define \prefix@section
\newcommand\prefix@section{Question \thesection}
\makeatother

\DeclareMathOperator{\rank}{rank}
\DeclareMathOperator{\lcm}{lcm}
\DeclareMathOperator{\tr}{Tr}

% Set your name here
\def\name{Problem Set 7}


\geometry{
  body={6.5in, 8.5in},
  left=1.0in,
  top=1.25in
}

% Customize page headers
\pagestyle{myheadings}
\markright{\name}
\thispagestyle{empty}

% Custom section fonts
\usepackage{sectsty}
\sectionfont{\rmfamily\mdseries\Large}
\subsectionfont{\rmfamily\mdseries\itshape\large}

% Other possible font commands include:
% \ttfamily for teletype,
% \sffamily for sans serif,
% \bfseries for bold,
% \scshape for small caps,
% \normalsize, \large, \Large, \LARGE sizes.

% Don't indent paragraphs.
\setlength\parindent{0em}

\begin{document}

{\huge \name}

% Alternatively, print name centered and bold:
%\centerline{\huge \bf \name}

\vspace{0.25in}

Dev Dabke \\
MATH 501: Introduction to Algebraic Structures I \\
October 27, 2016 \\
Prof.\ Calderbank \\

\section{}
\label{sec:Question1}

First, let us apply Sylow's theorem.
We know that $ 24 = 3 \times 2^3 $.
Let us find $ P_3 $ and $ P_2 $ and correspondingly, $ n_3 $ and $ n_2 $.
Note that $ n_3 | 2^3 $ and $ n_3 = 1 \mod{3} $, so can only be $ 1 $ or $ 4 $.
Similarly, $ n_2 | 3 $ and $ n_2 = 1 \mod{2} $ (this last condition does not add anything new), and thus can only be $ 1 $ or $ 3 $.
\\ \\
However, consider the case where $ n_2 = 1 $ or (an inclusive or, so both could be true, but we only need at least $ 1 $) $ n_3 = 1 $.
By Theorem 11.4, we see that we would have found the unique subgroup of that order and it would be normal.
In addition, these Sylow subgroups are cyclic since they have prime power order.
Without loss of generality, we will suppose $ n_3 = 1 $ and that $ P_3 $ is normal.
Take a subgroup $ S $ that is one of the Sylow $ 2 $-subgroups.
We can now generate a subgroup that is $ H = P_3 \times S $.
However, since $ P_3 $ is normal, we know that we end up with an element of order $ 6 $, which contradicts one of our assumptions.
We can apply this argument symmetrically to the case where $ n_2 = 1 $.
Thus, we know that we have $ n_3 = 4 $ and $ n_2 = 3 $.
\\ \\
Moving forward, we will invoke Lecture 9: Groups Acting on Sets.
Allow our group $ G $ to act on our Sylow $ 3 $-subgroups (by conjugation).
Beautifully, we know that this group action can only permute the Sylow $ 3 $-subgroups.
Therefore, this arises naturally to give us a surjective homomorphism $ \phi $ between $ G $ and $ S_4 $, namely:
\[
\phi : G \to S_4
\]
Now, denote $ x \in P_3 : |x| = 3 $ where $ P_3 $ is one of the Sylow $ 3 $-subgroups.
By Exam I, Question 2, Part ii, we actually know the structure of the conjugacy classes of the elements of order $ 3 $.
In particular, we know that all of the elements of order $ 3 $ are contained in the Sylow $ 3 $-subgroups and they together form a conjugacy class.
Note that this conjugacy class has $ 8 $ elements and is more generally an orbit.
Therefore, we know that stabilizer of any element is of order $ 3 $.
\\ \\
Now, for an element $ x \in P_3 $, we know that it commutes with all of the elements in $ P_3 $, since $ P_3 $ is cyclic, i.e.\ $ P_3 $ is in the stabilizer of $ x $.
Since $ |P_3| = 3 $ and the order of the stabilizer of $ x $ is also $ 3 $, $ P_3 $ must actually be the stabilizer.
Therefore, the stabilizer of any element is the Sylow $ 3 $-subgroup it comes from.
\\ \\
Furthermore, the stabilizer of the conjugacy class is the intersection of the stabilizer of each element.
The intersection of the stabilizers of each element is simply the intersection of the Sylow $ 3 $-subgroups, which we know is trivial.
Therefore, the stabilizer of the conjugacy class is trivial.
Since this is not just any stabilizer, but specifically the stabilizer of a conjugacy class, we have found some information about the normalizers of these elements.
In particular, the intersection of the normalizers of the Sylow $ 3 $-subgroups must be trivial.
\\ \\
Looking back to $ \phi $, we know that by definition, $ K $ must be contained in the intersection of the normalizers of the Sylow $ 3 $-subgroups.
But, as we just demonstrated, this intersection is trivial; thus $ K $ is trivial.
By the First Isomorphism Theorem in Lecture 5 (Theorem 5.11), we know that $ S_4 \cong G / K = G $.
Therefore, we know that any group $ G $ of order $ 24 $ with no element of order $ 6 $ is isomorphic to the symmetric group $ S_4 $.
\QED{}

\section{}
\label{sec:Question2}

Note that all of the non-zeros elements of the field form a cyclic, multiplicative group of order $ p^{m} - 1 $.
(I honestly have no clue why this is true, but it was in lecture, so bully for us!)
Pick arbitrary $ \alpha $ as a generator of this group.
Now, take the product $ \pi $ of all distinct group elements:
\[
\pi = \prod_{i = 1}^{p^{m} - 2} \alpha^i
\]
Note that $ \alpha \times \alpha^{p^{m} - 2} = p^{m} - 1 = 1 $.
In fact, we can pair off all of the elements on the ends to form the identity.
Since $ p $ is odd, we know that $ p^{m} $ is odd.
Thus $ p^{m} - 2 $ is odd as well.
Therefore, there will be one element in the middle that cannot pair off.
This element as order $ \frac{1 + (p^m - 2)}{2} = \frac{p^m - 1}{2} $.
As a sanity check, note that $ p^m - 1 $ is even, so this all makes sense.
Therefore, we know that, in fact $ \pi = \alpha^{\frac{p^m - 1}{2}} $.
But, if we square this element:
\[
\pi^2 = \left(\alpha^{\frac{p^m - 1}{2}} \right)^2 = \alpha^{p^m - 1} = 1
\]
Since $ \pi^2 = 1 $, we know that $ \pi = \pm 1 $.
However, if $ \pi = 1 $, then the order of $ \alpha $ would be $ \frac{p^m - 1}{2} $, a contradiction.
Therefore, $ \pi = -1 $.
\QED{}

\section{}
\label{sec:Question3}
Denote $ tr_t $ to be the ``term-wise trace'' for convenience.
\[
A =
tr_t \left(\begin{bmatrix}
    \alpha^2 & \alpha^3 & \alpha^4 \\
    \alpha^3 & \alpha^4 & \alpha^5 \\
    \alpha^4 & \alpha^5 & \alpha^6
\end{bmatrix} \right) =
\begin{bmatrix}
    0 & 1 & 0 \\
    1 & 0 & 1 \\
    0 & 1 & 1
\end{bmatrix}
\]
This was quite easy to find.
In particular, we note that $ \alpha^3 = \alpha + 1 $.
Therefore, $ \alpha \times \alpha^3 = \alpha^2 + \alpha $.
By Lemma 14.7, we know that the trace of $ \alpha^4 $ is $ 0 $.
We also know that the trace splits the powers of $ \alpha $ into $ 2 $, with half evaluating to $ 1 $ and the other half evaluating to $ 0 $.
And this how we found $ A $.
We can now easily form $ B = A^{-1} $.
\[
B =
\begin{bmatrix}
    1 & 1 & 1 \\
    1 & 0 & 0 \\
    1 & 0 & 1
\end{bmatrix}
\]
\\ \\
By rigorously applying the trace rules for the dual basis, we generate:
\begin{align}
    \beta_0 &= \alpha^2 + 1 \\
    \beta_1 &= \alpha \\
    \beta_2 &= 1
\end{align}
We can generate a matrix of coordinates:
\[
\Gamma =
\begin{bmatrix}
    1 & 0 & 1 \\
    0 & 1 & 0 \\
    1 & 0 & 0
\end{bmatrix}
\]
Thus, we complete our calculations.
\QED{}

\section{}
\label{sec:Question4}

For there to be a subgroup $ G $ of $ K* $ such that $ |G| = 24 $, by Lagrange's Theorem, $ |K*| = 24 \times n $ where $ n $ is some strictly positive integer.
However, we know that $ K* $ is cyclic and we can write $ K* = \langle x | x^{24 \times n} = 1 \rangle $.
Therefore, to generate $ G $, we only have one choice:
\[
\langle x^n \rangle
\]
since there are exactly $ 24 $ distinct arbitrary powers of $ x^n $ in this subgroup.
If we take an arbitrary power $ x^c $ where $ c $ does not divide the order of $ K* $, we would just regenerate $ K* $ again.
Otherwise, we would find another cyclic group.
This follows from the structure of cyclic groups and their respective subgroups.
\\ \\
Therefore, if $ G_1 $ and $ G_2 $ both have order $ 24 $, they must be generated by an element of $ \{x^n, {(x^n)}^2, \ldots, {(x^n)}^{23}\} $.
By the structure of cyclic groups (and subgroups), these generators generate equivalent groups (subgroups).
Thus $ G_1 = G_2 $.
\QED{}

\section{}
\label{sec:Question5}

Sadly, we will approach this via a dressed-up brute force.
I tried all of the factorizations that I say do not work.
\\ \\
Note that $ p_0(x) = x^5 + x^2 + 1 $ clearly has no linear factors, i.e.\ $ 0 $ or $ 1 $.
Therefore, it also cannot have factors of order $ 5 $.
$ x $ and $ x + 1 $ also do not divide $ p_0 $, which tell us that $ p_0 $ cannot have factors of order $ 4 $ either.
$ x^2 $, $ x^2 + x $, $ x^2 + 1 $, and $ x^2 + x + 1 $ also do not divide $ p_0 $ and so $ p_0 $ also has no factors of order $ 3 $ either.
Thus, $ p_0 $ is irreducible.
\\ \\
Let us try $ p_1(x) = x^6 + x + 1 $.
Again, no linear factors, so no factors of order $ 6 $.
It has no factors of order $ 1 $ ($ x $ or $ x + 1 $), so $ p_1 $ has no factors of order $ 5 $.
It has no factors of order $ 2 $ ($ x^2 $, $ x^2 + x $, $ x^2 + 1 $, and $ x^2 + x + 1 $), so it also has no factors of order $ 4 $.
Finally, $ x^3 $, $ x^3 + 1 $, $ x^3 + x $, $ x^3 + x^2 $, $ x^3 + x + 1 $, $ x^3 + x^2 + 1 $, $ x^3 + x^2 + x $, and $ x^3 + x^2 + x + 1 $ do not divide $ p_1 $, so it has no factors of order $ 3 $.
Therefore, $ p_1 $ is irreducible.
\\ \\
Finally, note that $ p_1(y^{-1}) = 1 + {(y^{-1})}^5 + {(y^{-1})}^6 = p_2(z) $ where $ z = y^{-1} $ and $ p_2(x) = x^6 + x^5 + 1 $.
Therefore, $ p_2 $ is also irreducible.
(As a sanity check, none of the previous factors that we tried for $ p_1 $ divide $ p_2 $.)
\QED{}

\section{}
\label{sec:Question6}

First, label $ q_1 = x^3 + x + 1 $ and $ q_2 = x^3 + x^2 + 1 $.
Note that $ q_1(x^{-1}) = q_2 $ and $ q_1 = q_2(x^{-1}) $.
Thus, if $ \alpha $ is a root of $ q_1 $, then $ \alpha^{-1} $ has to be a root of $ q_2 $ (and vice versa).
Since there is a bijective correspondence between elements and their inverses, we know that there is a bijective correspondence between the roots of $ q_1 $ and $ q_2 $.
In particular, we know we have three distinct elements: $ \alpha_0 = \alpha $, $ \alpha_1 = \alpha^2 $, and $ \alpha_2 = \alpha^3 = \alpha + 1 $ of $ q_1 $.
Allow $ \beta_0 $, $ \beta_1 $, $ \beta_2 $ to be the respective inverses that correspond to $ q_2 $.
Therefore, we know that any field automorphism $ \phi : K \to L $ is completely determined by $ \phi(\alpha) $.
In addition, $ \alpha $ could be mapped to any of the inverses, i.e.\ the three $ \beta $s.
Therefore, these are the three possible choices for $ \phi(\alpha) $.
\QED{}

\section{}
\label{sec:Question7}

\subsection{Part i}
\label{sec:7Parti}

This actually follows directly from the linearity of the trace.
We see that $ \tr{(\gamma)} = 0 \iff \gamma = 0 $.
Therefore, $ ax = 0 $ for every field element $ x $.
The only $ a $ with this property is $ 0 $ itself.
\QED{}

\subsection{Part ii}
\label{sec:7Partii}

By the linearity of the trace, we see that $ \tr{(x^2y + xy^2)} = \tr{(x^2 y)} + \tr{(xy^2)} $.
For $ \tr{(x^2y + xy^2)} = 0 $, we know that either $ \tr{(x^2 y)} = \tr{(xy^2)} = 0 $ or $ \tr{(x^2 y)} = \tr{(xy^2)} = 1 $.
Suppose the $ 0 $ case.
Quickly, by Part i, we see that $ \tr{(xy^2)} = 0 \implies y^2 = 0 $.
If $ y^2 = 0 $ then $ y = 0 $.
Since we are interested in the non-zero $ y $ values, we throw this out.
\\ \\
Thus, take $ \tr{(x^2 y)} = \tr{(xy^2)} = 1 $.
To solve these equations, if $ y = y^2 $ (i.e.\ they are not distinct) then $ y = 1 $.
Otherwise, we have \textit{at most} two distinct elements $ y $ and $ y^2 $ that correctly solve our equation.
Therefore, we have \textit{at most} $ 3 $ distinct, non-zero elements: $ 1 $, $ y $, and $ y^2 $.

\subsection{Part iii}
\label{sec:7Partiii}

Let us restate our problem in terms of what we have learned from Part ii.
First, we see that where $ y = 1 $, this condition is always satisfied, regardless of the parity of $ m $.
Next, we want to show that $ y = y^2 $ if and only if $ m $ is odd; in addition, we want to show that there exists a distinct $ y : y \neq y^2 $ that satisfies our trace equations in Part ii if and only if $ m $ is even.
\\ \\
First, note that only $ 2^{m-1} $ elements evaluate to $ 1 $ under the trace function.
If $ m $ is odd, then $ m - 1 $ is even.
However, note that if $ m - 1 $ is even, we can find integer power $ \frac{m - 1}{2} $.
In that case, we can find $ y = \lambda^{\frac{m - 1}{2}} $.
But then, this implies that $ y^2 = \lambda^{m - 1} $.
Therefore, we can see that $ y $ and $ y^2 $ are not distinct.
Furthermore, where $ y = y^2 $, we know that the only element for which this is true is $ 1 $ (note that in $ \mathbb{F}_2 $, $ \pm 1 = 1 $).
\\ \\
When $ m $ is even, then $ m - 1 $ is odd.
Therefore, we cannot find a corresponding integer power $ \frac{m - 1}{2} $, and thus there exists a solution $ y $ and $ y^2 $ such that $ y \neq y^2 $.
\QED{}

\end{document}
