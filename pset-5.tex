\documentclass[letterpaper]{article}

\usepackage{amsmath}
\usepackage{amssymb}
\usepackage{hyperref}
\usepackage{geometry}
\usepackage{nth}

% Comment the following lines to use the default Computer Modern font
% instead of the Palatino font provided by the mathpazo package.
% Remove the 'osf' bit if you don't like the old style figures.
\usepackage[T1]{fontenc}
\usepackage[sc,osf]{mathpazo}

\newcommand*{\QED}{\hfill\ensuremath{\square}}%

\DeclareMathOperator{\rank}{rank}

% Set your name here
\def\name{Problem Set 5}


\geometry{
  body={6.5in, 8.5in},
  left=1.0in,
  top=1.25in
}

% Customize page headers
\pagestyle{myheadings}
\markright{\name}
\thispagestyle{empty}

% Custom section fonts
\usepackage{sectsty}
\sectionfont{\rmfamily\mdseries\Large}
\subsectionfont{\rmfamily\mdseries\itshape\large}

% Other possible font commands include:
% \ttfamily for teletype,
% \sffamily for sans serif,
% \bfseries for bold,
% \scshape for small caps,
% \normalsize, \large, \Large, \LARGE sizes.

% Don't indent paragraphs.
\setlength\parindent{0em}

\begin{document}

{\huge \name}

% Alternatively, print name centered and bold:
%\centerline{\huge \bf \name}

\vspace{0.25in}

Dev Dabke \\
MATH 501: Introduction to Algebraic Structures I \\
October 4, 2016 \\
Prof.\ Calderbank \\

\section{Question 1}
\label{sec:Question1}

First, let us develop some useful labels for our 4-by-4 matrices:

\begin{align*}
    A &=
    \begin{bmatrix} + & & & \\ & + & & \\ & & + & \\ & & & + \end{bmatrix}
    \\
    B &=
    \begin{bmatrix} & + & & \\ - & & & \\ & & & - \\ & & + & \end{bmatrix}
    \\
    C &= \begin{bmatrix} & & - & \\ & & & - \\ + & & & \\ & + & & \end{bmatrix}
    \\
    D &=
    \begin{bmatrix} & & & - \\ & & + & \\ & - & & \\ + & & & \end{bmatrix}
    \\
\end{align*}

Next, we will take elements of $ D_4 $:

\begin{align*}
    I_2 &= \begin{bmatrix} + & \\ & + \end{bmatrix}
    \\
    x &= \begin{bmatrix} & + \\ + & \end{bmatrix}
    \\
    xz &= \begin{bmatrix} & - \\ + & \end{bmatrix}
    \\
    xzx &= \begin{bmatrix} - & \\ & + \end{bmatrix}
\end{align*}

Thus, we can write that:
\begin{align}
    A &= I_2 \otimes I_2 \label{eqA} \\
    B &= xzx \otimes xz \label{eqB} \\
    C &= x \otimes xz \label{eqC} \\
    D &= x \otimes xzx \label{eqD}
\end{align}

\section{Question 2}
\label{sec:Question2}

We will prove this by induction.
Our inductive hypothesis is:
\[
    e_v = e_{v_{m-1}} \otimes \ldots \otimes e_{v_{0}}
\]

As a base case, we will prove the $ m = 0 $ case (a very simple case).
Here, $ v = (v_0) $.
$ v_0 $ can either be $ 0 $ or $ 1 $.
If $ v_0 = 0 $, then $ e_{v_0} = e_0 = (1,0) $.
Trivially, $ e_v = e_{(v_0)} = (1,0) $.
If $ v_0 = 1 $, then $ e_{v_0} = e_1 = (0,1) $.
Trivially, $ e_v = e_{(v_0)} = (0,1) $.
Thus, where $ m = 0 $, $ e_v = e_{v_0} $.
\\ \\
Now, assuming the $ m $ case, we will prove the $ m + 1 $ case.
Construct a binary, $ (m + 1) $-tuple $ w $, i.e.\ $ w = (w_m, w_{m-1}, \ldots, w_0) $.
Construct $ u : u = (w_{m-1}, w_{m-2}, \ldots, w_0) $.
For $ u $, by our inductive hypothesis, we know that:
$$ e_u = e_{w_{m-1}} \otimes \ldots \otimes e_{w_{0}} $$

We will now construct a useful vector $ x : x = e_{w_m} \otimes e_u $.
We will show that $ e_w = x $, which in turn will demonstrate that $ e_w = e_{w_m} \otimes e_{w_{m-1}} \otimes \ldots \otimes e_{w_{0}} $.
Conveniently, $ w_m $ can only have two values $ 0 $ or $ 1 $.
Let us investigate what happens in each case.
\\ \\
Where $ w_m = 0 $, we see that $ e_{w_m} = e_0 = (1,0) $.
Thus $ x = (1,0) \otimes e_u $, which looks like $ x = (e_u, \vec{0}) $, where $ \vec{0} $ is the vector of all $ 0 $ entries with the same length as $ e_u $.
In short, the first $ 2^{m-1} $ terms of $ x $ are the terms of $ e_u $ and the next $ 2^{m-1} $ terms are all $ 0 $.
Note that we have constructed a vector with length $ 2^{m-1} + 2^{m-1} = 2(2^{m-1}) = 2^1 2^{m-1} = 2^{(1 + m - 1)} = 2^m $.
We can finally see what is actually going on:
we are taking a binary number and adding a new bit; in particular, we are adding a most-significant digit (bit).
\\ \\
Therefore, in this case, we have added an extra $ 0 $ to our number on the left side, i.e.\ as the most-significant digit (bit).
An arbitrary number $ k $ written out with $ m $ digits that has a $ 0 $ added as the most-significant digit (bit) does not change value.
Therefore, when $ w_m $ is $ 0 $, we do not change the value of what this number represents from the value that $ u $ represents.
In short, the representation vector is simply twice as long; we copy the original part of the vector, and fill the rest of the bits with $ 0 $s, as the $ 1 $ will already be in the right position.
This would look like $ x = (e_u, \vec{0}) $, which is exactly what we get.
If we inspect $ e_w $, we see that we get a vector with a $ 1 $ in the same spot as $ e_u $, just with twice the length.
Therefore, where $ w_m = 0 $, we see that $ e_w = x = e_{w_m} \otimes e_u $.
\\ \\
Similarly, where $ w_m = 1 $, we see that $ e_{w_m} = e_1 = (0,1) $.
We yield $ x = (0,1) \otimes e_u $, which looks like $ x = (\vec{0}, e_u) $, where $ \vec{0} $ is as it was before.
We again have constructed a vector of length $ 2^m $.
Let us once more consider the argument by looking at what $ w_m $ represents.
If it is a $ 1 $, we are constructing a new number with $ 1 $ as its most-significant digit (bit).
Designate $ y $ as the number that $ u $ encodes in binary.
It is in the $ m $ position, so our new number $ 2^{m - 1} + y $.
In short, in our $ 2^m $ representation vector, we have to shift over wherever our $ 1 $ is by $ 2^{m - 1} $ spots to add the value $ 2^{m - 1} $.
Note that this is not a special process for powers of two.
In fact, for any representation vector, to add any value $ z $ to our current representation, we would just need to move our $ 1 $ over by $ z $ spots.
\\ \\
What would this new representation vector look like with our $ 1 $ shifted over by $ 2^{m-1} $?
It would look like $ x = (\vec{0}, e_u) $!
Looking at $ e_w $, we have correctly shifted over the $ 1 $ in the correct spot, which would $ 2^{m - 1} $ places over to the right.
Again, $ e_w = x = e_{w_m} \otimes e_u $
\\ \\
Looking across the quad to the ECE department, we see that we have in fact constructed a decoder.
Since we have that $ e_w = e_{w_m} \otimes e_u $ no matter the value of $ w_m $, we can make our formal conclusion by substitution.
Since $ e_u = e_{w_{m-1}} \otimes \ldots \otimes e_{w_{0}} $, we see that:
$$ e_w = e_{w_m} \otimes e_{w_{m-1}} \otimes \ldots \otimes e_{w_{0}} $$
\QED{}

\section{Question 3}
\label{sec:Question3}

We will prove this by exhaustion.
\\ \\
First, we are going to inspect $ e_{v_{i} + a_{i}} $ and $ e_{v_{i}} \begin{bmatrix} & + \\ + & \end{bmatrix}^{a_{i}} $.
In particular, we will show that:
\[
e_{v_{i} + a_{i}} = e_{v_{i}} \begin{bmatrix} & + \\ + & \end{bmatrix}^{a_{i}}
\]
To start, we will need to have a few facts at our disposal:
\begin{align}
    0 + 0 &= 1 + 1 &= 0 \\
    0 + 1 &= 1 + 0 &= 1
\end{align}
and:
\begin{align}
    \begin{bmatrix} & + \\ + & \end{bmatrix}^{0} &= \begin{bmatrix} + & \\ & + \end{bmatrix} \\
    \begin{bmatrix} & + \\ + & \end{bmatrix}^{1} &= \begin{bmatrix} & + \\ + & \end{bmatrix}
\end{align}

Now, we know that $ v_{i} $ and $ a_{i} $ can each either be $ 0 $ or $ 1 $.
We will also note $ x^1 = \begin{bmatrix} & + \\ + & \end{bmatrix} $ and $ x^0 = \begin{bmatrix} + & \\ & + \end{bmatrix} $
Let us enumerate all of the possibilities:
\begin{center}
\begin{tabular}{c | c | c | c }
\hline
$ v_i $ & $ a_i $ & $ e_{v_i + a_i} $ & $ e_{v_i} x^{a_i} $ \\
\hline
\hline
 0 & 0 & $ e_0 = (1, 0) $ & $ (1, 0) x^0 = (1, 0) $ \\
 0 & 1 & $ e_1 = (0, 1) $ & $ (1, 0) x^1 = (0, 1) $ \\
 1 & 0 & $ e_1 = (0, 1) $ & $ (0, 1) x^0 = (0, 1) $ \\
 1 & 1 & $ e_0 = (1, 0) $ & $ (0, 1) x^1 = (1, 0) $

\end{tabular}
\end{center}

As we can see, we have proven that
\begin{equation}
    \label{eqn:eviadao}
    e_{v_{i} + a_{i}} = e_{v_{i}} \begin{bmatrix} & + \\ + & \end{bmatrix}^{a_{i}}
\end{equation}
\\ \\
Next, we will introduce a fact about the Kronecker product for matrices:
\begin{equation}
(A \otimes B) (C \otimes D) = AC \otimes BD
\label{eqn:kronecker-product}
\end{equation}
where the products $ AC $, $ BD $, and $ (A \otimes B) (C \otimes D) $ are well-defined.
\\ \\
Now, we can finally tie all of this together to yield our proof.
First, let us consider the structure of $ e_{v + a} $.
For convenience, let us write $ v = (v_0, \ldots, v_{m - 1}) $ and $ a = (a_0, \ldots, a_{m - 1}) $.
We see that $ v + a = (v_0 + a_0, \ldots, v_{m - 1} + a_{m - 1}) $.
By our proof for Question~\ref{sec:Question2}, then
\begin{equation}
    \label{eqn:evplusa}
e_{v + a} = e_{v_{m - 1} + a_{m - 1}} \otimes \ldots \otimes e_{v_0 + a_0}
\end{equation}
\\ \\
If we look at $ e_v D(a, 0) $, we yield something similar.
Since $ e_v = e_{v_{m - 1}} \otimes \ldots \otimes e_{v_0} $ (by Question~\ref{sec:Question2}) and (by definition) $ D(a, 0) = x^{a_{m - 1}} \otimes \ldots \otimes x^{a_{0}} $, we can use the fact in Equation~\ref{eqn:kronecker-product} and the associativity of the Kronecker product to see that
\begin{equation}
    \label{eqn:evdao}
    e_v D(a, 0) = (e_{v_{m - 1}} \otimes \ldots \otimes e_{v_0})(x^{a_{m - 1}} \otimes \ldots \otimes x^{a_{0}}) = e_{v_{m - 1}} x^{a_{m - 1}} \otimes \ldots \otimes e_{v_0} x^{a_{0}}
\end{equation}
\\ \\
Finally, let us combine Equation~\ref{eqn:eviadao}, Equation~\ref{eqn:evplusa}, and Equation~\ref{eqn:evdao} to formally conclude our proof.
\begin{align*}
    e_{v + a} &= e_{v_{m - 1} + a_{m - 1}} \otimes \ldots \otimes e_{v_0 + a_0} &\text{by Equation~\ref{eqn:evplusa}} \\
    e_{v + a} &= e_{v_{m - 1}} x^{a_{m - 1}} \otimes \ldots \otimes e_{v_0} x^{a_{0}} &\text{substitution by Equation~\ref{eqn:eviadao}} \\
    e_{v + a} &= e_v D(a, 0) &\text{by Equation~\ref{eqn:evdao}}
\end{align*}
\QED{}

\section{Question 4}
\label{sec:Question4}

This proof will follow a similar train of though as Question~\ref{sec:Question3}.
\\ \\
Let us look at the product $ e_v D(0, b) $ (using Question~\ref{sec:Question2})
\[
e_v D(0, b) = (e_{v_{m - q}} \otimes \ldots \otimes e_{v_0})(z^{b_{m - 1}} \otimes \ldots \otimes z^{b_{0}})
\]

Recall the convenient property in Equation~\ref{eqn:kronecker-product}:
\begin{equation}
(A \otimes B) (C \otimes D) = AC \otimes BD
\tag{\ref{eqn:kronecker-product} revisited}
\end{equation}
By this property:
\[
e_v D(0, b) = (e_{v_{m - q}} \otimes \ldots \otimes e_{v_0})(z^{b_{m - 1}} \otimes \ldots \otimes z^{b_{0}}) = e_{v_{m - q}} z^{b_{m - 1}} \otimes \ldots \otimes e_{v_0}z^{b_{0}}
\]

We will also note $ z^1 = \begin{bmatrix} + & \\ & - \end{bmatrix} $ and $ z^0 = \begin{bmatrix} + & \\ & + \end{bmatrix} $
\\ \\
We will now use another ``truth table'' to exhaustively show what happens with the different possible values of $ v_i $ and $ b_i $ respectively:

\begin{center}
\begin{tabular}{c | c | c | c }
\hline
$ v_i $ & $ b_i $ & $ e_{v_i} $ & $ e_{v_i} z^{b_i} $ \\
\hline
\hline
 0 & 0 & $ e_0 = (1, 0) $ & $ (1, 0) z^0 = (1, 0) = e_0 $ \\
 0 & 1 & $ e_1 = (0, 1) $ & $ (1, 0) z^1 = (1, 0) = e_0 $ \\
 1 & 0 & $ e_1 = (0, 1) $ & $ (0, 1) z^0 = (0, 1) = e_1 $ \\
 1 & 1 & $ e_0 = (1, 0) $ & $ (0, 1) z^1 = (0, -1) = -e_1 $

\end{tabular}
\end{center}

Interestingly, we see that $ e_{v_i} z^{b_i} = e_{v_i} $ \textit{unless} $ v_i = b_i = 1 $.
In the case that $ v_i = b_i = 1 $, we see that $ e_{v_i} z^{b_i} = -e_{v_i} $
Let us say that $ \delta_i = 1 \iff v_i = b_i = 1 $ and $ \delta_i = 0 $ otherwise.
Now, we have certain properties of this binary multiplication, namely:
\begin{align*}
    (0)(0) &= 0 \\
    (0)(1) &= 0 \\
    (1)(0) &= 0 \\
    (1)(1) &= 1
\end{align*}

Fortunately, this aligns with exactly the function we hoped to describe with $ \delta_i $.
Therefore, we write $ \delta_i = (v_i)(b_i) $
\\ \\
Thus, we can finally equate the terms of $ e_v $ with the terms in the form of $ e_{v_i} z^{b_i} $.
Namely:
\[
e_{v_i} z^{b_i} = {(-1)}^{\delta_i} e_{v_i}
\]

Therefore, by substitution:
\begin{equation}
\label{eqn:q4subs}
e_v D(0, b) = e_{v_{m - 1}} z^{b_{m - 1}} \otimes \ldots \otimes e_{v_0}z^{b_{0}} = {(-1)}^{\delta_{m - 1}} e_{v_{m - q}} \otimes \ldots \otimes {(-1)}^{\delta_0} e_{v_0}
\end{equation}

Fortunately, we can now invoke another property of the Kronecker product:
\begin{equation}
\label{eqn:kronecker-scalar}
    (cA) \otimes B = A \otimes (cB) = c(A \otimes B)
\end{equation}
where $ c $ is some scalar.
Thus, we can ``collect'' the $ \delta $ terms all onto the left side of the Kronecker product:
\[
 {(-1)}^{\delta_{m - 1}} e_{v_{m - q}} \otimes \ldots \otimes {(-1)}^{\delta_0} e_{v_0} = \left[{(-1)}^{\delta_{m - 1}} \ldots {(-1)}^{\delta_0} \right] \left(e_{v_{m - q}} \otimes \ldots \otimes e_{v_0} \right)
\]
\\ \\
We will now work with the $ \delta $ terms.
In particular, we will aggregate the exponents:
\[
{(-1)}^{\delta_{m - 1}} \ldots {(-1)}^{\delta_0} = {(-1)}^{\delta_{m - 1} + \cdots + \delta_0}
\]
Recall our definition of $ d_i $, in that $ d_i = (v_i)(b_i) $.
Thus, $ \delta_{m - 1} + \cdots + \delta_0 = (v_{m - 1})(b_{m - 1}) + \cdots + (v_0)(b_0) $.
By definition $ b v^T = (v_{m - 1})(b_{m - 1}) + \cdots + (v_0)(b_0) $.
Combining these two equations yield
\[
{(-1)}^{\delta_{m - 1} + \cdots + \delta_0} = {(-1)}^{b v^T}
\]
Finally, we put everything together:
\begin{equation}
\label{eqn:q4gather}
    \left[{(-1)}^{\delta_{m - 1}} \ldots {(-1)}^{\delta_0} \right] \left(e_{v_{m - q}} \otimes \ldots \otimes e_{v_0} \right) = {(-1)}^{b v^T} \left(e_{v_{m - q}} \otimes \ldots \otimes e_{v_0} \right)
\end{equation}

We can now formally complete the proof
We combine Equation~\ref{eqn:q4subs} and Equation~\ref{eqn:q4gather} to see that:
\[
e_v D(0, b) = e_{v_{m - 1}} z^{b_{m - 1}} \otimes \ldots \otimes e_{v_0}z^{b_{0}} = {(-1)}^{\delta_{m - 1}} e_{v_{m - q}} \otimes \ldots \otimes {(-1)}^{\delta_0} e_{v_0} = {(-1)}^{b v^T} \left(e_{v_{m - q}} \otimes \ldots \otimes e_{v_0} \right)
\]
Since $ e_v = \left(e_{v_{m - q}} \otimes \ldots \otimes e_{v_0} \right) $ proved in Question~\ref{sec:Question2}, we can conclude that:
\[
e_v D(0, b) = {(-1)}^{b v^T} e_v
\]
\QED{}

\section{Question 5}
\label{sec:Question5}

\subsection{Part i}
\label{sub:5Parti}

To show that $ G $ is a group, we have to show that
\begin{enumerate}
    \item the operator is associative
    \item that the operator permits an inverse
    \item that the group is closed (to be pedantic, that the group operator has the same domain and codomain, namely the set of matrices)
\end{enumerate}

First, we know that the operator is associative since matrix multiplication is associative.
Second, we will prove that the operator permits a well-defined inverse.
Note that an element $ A = D(a,b) = D(a,0)D(0,b) $ ultimately looks like:
\[
A = D(a,0)D(0,b) = D(a,b) = (x^{a_{m - 1}} \otimes \ldots \otimes x^{a_{0}}) (z^{b_{m - 1}} \otimes \ldots \otimes z^{b_{0}}) = (x^{a_{m - 1}} z^{b_{m - 1}}) \otimes \ldots \otimes  (x^{a_{0}} z^{b_{0}})
\]
As a result from linear algebra, we know that the product of two non-singular matrices is also non-singular.
Since $ x $ and $ z $ are non-singular and $ x^0 = z^0 = I_2 $ is non-singular, we see that every term $ x^{a_i} z^{b_i} $ is also non-singular and that $ \rank{(x)} = \rank{(z)} = 2 $
More precisely, $ k_i = x^{a_i} z^{b_i} $ has rank $ 2 $, since it is a non-singular $ 2 $-by-$ 2 $ matrix.
We will now invoke a useful property of the Kronecker product, namely that:
$$ \rank{(C \otimes D)} = \rank{(C)}\rank{(D)} $$
Therefore, we see that $ A $ has rank $ 2m $, since it is the Kronecker product of $ m $ matrices with rank $ 2 $.
Since it is a $ 2m $-by-$ 2m $ matrix, and thus square, and has full rank, it is non-singular.
Since it is non-singular, it has an inverse.
Thus, the operator permits a well-defined inverse.
\\ \\
Finally, we need to show that the group is closed.
Recall the property of the Kronecker product in Equation~\ref{eqn:kronecker-product}:
\begin{equation}
(A \otimes B) (C \otimes D) = AC \otimes BD
\tag{\ref{eqn:kronecker-product} revisited}
\end{equation}
What we want to show is that the group operator (matrix multiplication) is closed, i.e.\ if we multiply two elements of the group, we will get a result also in the group.
Let us take another generic element $ B \in G : B = D(a', b') $ and take its product with $ A $:
\[
AB = D(a,b)D(a',b') = \left[(x^{a_{m - 1}} z^{b_{m - 1}}) \otimes \ldots \otimes (x^{a_{0}} z^{b_{0}}) \right] \left[(x^{a'_{m - 1}} z^{b'_{m - 1}}) \otimes \ldots \otimes  (x^{a'_{0}} z^{b'_{0}}) \right]
\]
Using our handy property in Equation~\ref{eqn:kronecker-product}:
\begin{equation}
\label{eqn:kron-ab-expansion}
    \left[(x^{a_{m - 1}} z^{b_{m - 1}}) \otimes \ldots \otimes (x^{a_{0}} z^{b_{0}}) \right] \left[(x^{a'_{m - 1}} z^{b'_{m - 1}}) \otimes \ldots \otimes  (x^{a'_{0}} z^{b'_{0}}) \right] = (x^{a_{m - 1}} z^{b_{m - 1}} x^{a'_{m - 1}} z^{b'_{m - 1}}) \otimes \ldots \otimes (x^{a_{0}} z^{b_{0}} x^{a'_{0}} z^{b'_{0}})
\end{equation}

Now, we are allowed to have anything that looks like $ \pm D(a, b) $, so we will proceed accordingly.
What we will demonstrate is that:
$ x^r z^s x^t z^u = {(-1)}^k x^v z^w $
for some $ k $.
\\ \\
Conveniently, we see that $ r $, $ s $, $ t $, and $ u $ must be either $ 0 $ or $ 1 $.
If $ s = 0 $, since this produces the identity, we generate something in the form:
\[
x^r x^t z^u = x^{r + t} z^u
\]
so, $ v = r + t $, $ u = w $, and $ k = 0 $; our equality is satisfied.
Similarly, if $ t = 0 $, then $ w = s + u $ and $ v = r $ ($ k = 0 $).
If $ s = t = 0 $, then $ v = r $ and $ w = u $ ($ k = 0 $).
\\ \\
Now, we will use the fact that $ xz = -zx $ (i.e.\ these element anti-commute).
If $ r = u = 0 $, and $ s = t = 1 $, then we see that we directly use the property of anti-commuting, in that:
\[
z^1 x^1 = (-1) x^1 z^1
\]
Now, assume either $ r = 0 $ or $ u = 0 $.
In this case, we can ``move'' the $ z $ or $ x $ terms together, with a ``sign penalty,'' i.e.\ $ k = 1 $.
Now, for every term $ (x^{a_{i}} z^{b_{i}} x^{a'_{i}} z^{b'_{i}}) $, we have that:
\[
(x^{a_{i}} z^{b_{i}} x^{a'_{i}} z^{b'_{i}}) = {(-1)}^{k_i} x^{c_i} z^{d_i}
\]
Putting this all together, where $ K $ is the sum of the $ k_i $ terms:
\[
AB = (x^{a_{m - 1}} z^{b_{m - 1}} x^{a'_{m - 1}} z^{b'_{m - 1}}) \otimes \ldots \otimes (x^{a_{0}} z^{b_{0}} x^{a'_{0}} z^{b'_{0}}) = {(-1)}^{K} D(c, d)
\]
Since $ {(-1)}^{K} D(c, d) $ is $ \pm D(c, d) $, we have shown that the product $ AB $ is indeed in the group.
Thus, we have closure.
\\ \\
Because we have elucidated these three properties, we know that we do indeed have a group with our group operator as matrix multiplication.
\\ \\
Finally, we will demonstrate that
\[
D(a,b) D(a', b') = {(-1)}^{a'b^T + b'a^T} D(a', b') D(a, b)
\]
Let us inspect a particular term $ x^{a_i} z^{b_i} x^{a'_i} z^{b'_i} $.
We are going to ``move'' the $ x^{a'_i} $ ``across'' the $ z^{b_i} $ term.
If $ a'_i = 0 $ or $ b_i = 0 $, this is trivial since at least one of the terms becomes an identity.
If $ a'_i = 1 $ and $ b_i = 1 $, then we have to do some work.
Since $ xz = -zx $ (they anti-commute), we know that:
$$ x^{a_i} zx z^{b'_i} = x^{a_i} x(-1)z z^{b'_i} = (-1) x^{a_i} xz z^{b'_i} $$
We can summarize all of this as:
$$ x^{a_i} z^{b_i} x^{a'_i} z^{b'_i} = \left[{(-1)}^{a'_i b_i} \right] x^{a_i} x^{a'_i} z^{b_i} z^{b'_i} $$
Fortunately, powers of $ x $ commute with each other as do powers of $ z $.
Thus:
\[
\left[{(-1)}^{a'_i b_i} \right] x^{a_i} x^{a'_i} z^{b_i} z^{b'_i} =
\left[{(-1)}^{a'_i b_i} \right] x^{a'_i} x^{a_i} z^{b'_i} z^{b_i}
\]

Now we will ``move'' the $ z^{b'_i} $ ``across'' the $ x^{a_i} $.
Symmetrically, we find that:
\[
x^{a'_i} x^{a_i} z^{b'_i} z^{b_i} = \left[{(-1)}^{b'_i a_i}\right] x^{a'_i} z^{b'_i} x^{a_i} z^{b_i}
\]
Putting this all together, we yield that:
\[
x^{a_i} z^{b_i} x^{a'_i} z^{b'_i} = \left[{(-1)}^{a'_i b_i} \right] \left[{(-1)}^{b'_i a_i}\right] x^{a'_i} z^{b'_i} x^{a_i} z^{b_i}
\]
and finally:
\[
x^{a_i} z^{b_i} x^{a'_i} z^{b'_i} = \left[{(-1)}^{a'_i b_i + b'_i a_i} \right] x^{a'_i} z^{b'_i} x^{a_i} z^{b_i}
\]
For convenience, denote $ \delta_i = a'_i b_i + b'_i a_i $.

We move back to the general terms:
\[
AB = (x^{a_{m - 1}} z^{b_{m - 1}} x^{a'_{m - 1}} z^{b'_{m - 1}}) \otimes \ldots \otimes (x^{a_{0}} z^{b_{0}} x^{a'_{0}} z^{b'_{0}})
\]
We substitute in what we have learned to yield:
\[
(x^{a_{m - 1}} z^{b_{m - 1}} x^{a'_{m - 1}} z^{b'_{m - 1}}) \otimes \ldots \otimes (x^{a_{0}} z^{b_{0}} x^{a'_{0}} z^{b'_{0}}) = ({(-1)}^{\delta_{m - 1}} x^{a_{m - 1}} z^{b_{m - 1}} x^{a'_{m - 1}} z^{b'_{m - 1}}) \otimes \ldots \otimes ({(-1)}^{\delta_{0}} x^{a_{0}} z^{b_{0}} x^{a'_{0}} z^{b'_{0}})
\]
By Equation~\ref{eqn:kronecker-scalar} (the scalar multiplication property of the Kronecker product), we can ``aggregate'' the $ \delta $ terms.
Therefore:
\begin{multline}
\label{eqn:q5pilong}
 AB = ({(-1)}^{\delta_{m - 1}} x^{a_{m - 1}} z^{b_{m - 1}} x^{a'_{m - 1}} z^{b'_{m - 1}}) \otimes \ldots \otimes ({(-1)}^{\delta_{0}} x^{a_{0}} z^{b_{0}} x^{a'_{0}} z^{b'_{0}})
 = \\
  \left[{(-1)}^{\delta_{m-1}} \ldots {(-1)}^{\delta_{0}}\right] \left[(x^{a_{m - 1}} z^{b_{m - 1}} x^{a'_{m - 1}} z^{b'_{m - 1}}) \otimes \ldots \otimes (x^{a_{0}} z^{b_{0}} x^{a'_{0}} z^{b'_{0}}) \right]
\end{multline}
Now let us focus on the $ \delta $ terms:
\[
{(-1)}^{\delta_{m-1}} \ldots {(-1)}^{\delta_{0}}
=
{(-1)}^{\delta_{m-1} + \cdots + \delta_{0}}
\]
Recalling our definition of $ \delta_i = a'_i b_i + b'_i a_i $.
Using a simple result from linear algebra, we see that:
$$ \delta_{m - 1} + \cdots + \delta = a'b^T + b'a^T $$
Thus:
\[
{(-1)}^{\delta_{m-1}} \ldots {(-1)}^{\delta_{0}}
=
{(-1)}^{\delta_{m-1} + \cdots + \delta_{0}}
=
{(-1)}^{a'b^T + b'a^T}
\]
We can substitute this result back into Equation~\ref{eqn:q5pilong}, we see that:
\[
AB = \left[{(-1)}^{a'b^T + b'a^T}\right]\left[(x^{a_{m - 1}} z^{b_{m - 1}} x^{a'_{m - 1}} z^{b'_{m - 1}}) \otimes \ldots \otimes (x^{a_{0}} z^{b_{0}} x^{a'_{0}} z^{b'_{0}}) \right]
\]

By symmetry, we see that the last expression enclosed in brackets is actually just $ BA $, i.e.
\[
BA = \left[(x^{a_{m - 1}} z^{b_{m - 1}} x^{a'_{m - 1}} z^{b'_{m - 1}}) \otimes \ldots \otimes (x^{a_{0}} z^{b_{0}} x^{a'_{0}} z^{b'_{0}}) \right]
\]

Thus, we put this all together and yield:
\[
AB = \left[{(-1)}^{a'b^T + b'a^T}\right] BA
\]
or equivalently by plugging in the definitions of $ B $ and $ A $:
\[
D(a,b)D(a',b') = \left[{(-1)}^{a'b^T + b'a^T}\right] D(a',b')D(a,b)
\]

\QED{}

\subsection{Part ii}
\label{sub:5Partii}

We are going to see what happens when we square $ A = D(a,b) $, some non-identity element.
Jumping to our Kronecker product definition defined in Equation~\ref{eqn:kron-ab-expansion}, we see that:
\[
A^2 = AA = (x^{a_{m - 1}} z^{b_{m - 1}} x^{a_{m - 1}} z^{b_{m - 1}}) \otimes \ldots \otimes (x^{a_{0}} z^{b_{0}} x^{a_{0}} z^{b_{0}})
\]
Looking at a particular element $ x^{a_{i}} z^{b_{i}} x^{a_{i}} z^{b_{i}} $, we will ``inner'' terms by the same logic as before:
\[
x^{a_{i}} z^{b_{i}} x^{a_{i}} z^{b_{i}}
=
{(-1)}^{a_i b_i} x^{a_{i}} x^{a_{i}} z^{b_{i}} z^{b_{i}}
\]
We will now return to the Kronecker products of all the terms and again ``aggregate'' the ones.
Since we have now performed this operation twice on this problem set already, the details are extremely clear.
Thus, we will simply move from result to result:
\[
A^2
=
(x^{a_{m - 1}} z^{b_{m - 1}} x^{a_{m - 1}} z^{b_{m - 1}}) \otimes \ldots \otimes (x^{a_{0}} z^{b_{0}} x^{a_{0}} z^{b_{0}})
=
{(-1)}^{\sum_i a_i b_i}
(x^{a_{m - 1}} x^{a_{m - 1}} z^{b_{m - 1}} z^{b_{m - 1}}) \otimes \ldots \otimes (x^{a_{0}} x^{a_{0}} z^{b_{0}} z^{b_{0}})
\]
Note that $ \sum_i a_i b_i = ab^T $.
Thus:
\[
A^2 =
{(-1)}^{ab^T}
(x^{a_{m - 1}} x^{a_{m - 1}} z^{b_{m - 1}} z^{b_{m - 1}}) \otimes \ldots \otimes (x^{a_{0}} x^{a_{0}} z^{b_{0}} z^{b_{0}})
\]
Interestingly, we see that every term is now in the form $ x^l x^l z^k z^k $.
Since $ z^2 = x^2 = I_2 $, i.e.\ $ |x| = |z| = 2 $, we see that $ x^l x^l z^k z^k = x^{2l} z^{2k} = (I_2)(I_2) = I_2 $.
Thus, we rewrite $ A^2 $ to be:
\[
{(-1)}^{ab^T}
(I_2) \otimes \ldots \otimes (I_2)
\]
Now note that $ (I_2) \otimes \ldots \otimes (I_2) = I_{2m} $.
Thus, we see that $ A^2 = {(-1)}^{ab^T} I_{2m} $, which is very nice indeed.
$ ab^T $ can either be $ 0 $ or $ 1 $, thus we see that where it is $ 0 $, $ A^2 = I_{2m} $, which makes $ |A| = 2 $.
If $ ab^T = 1 $, then $ A^2 = -I_{2m} $.
In this case, note that $ A^3 = A^2 A = -A $.
Thus, we know that the order of $ A $ cannot be $ 3 $.
But, as is obvious, $ A^2 = -I_{2m} \implies A^4 = I_{2m} $.
Thus, for a non-identity element $ A $, it must have either order $ 2 $ or order $ 4 $.
\QED{}

\section{Question 6}
\label{sec:Question6}
Note that the first six rounds will help us determine the actual bit sequence, while the \nth{7} round will simply tell us if we are working with the positive Walsh function or the negative one.
Thus, without loss of generality, we will assume we are working with only the positive Walsh functions, and we will return to the parity issue later.
\\ \\
First, pick some fixed $ a \in \mathbb{F}_2^6 $.
This vector $ a $ characterizes the Walsh function $ W_a $ of the Grand Prize Door.
Now, every round of the first six rounds proceeds as follows:
\begin{enumerate}
    \item The host picks some vector $ b \in \mathbb{F}_2^6 $
    \item We pick some vector $ c \in \mathbb{F}_2^6 $
    \item The host tells us if $ W_a(b) = W_a(c) $
\end{enumerate}
Let us enumerate what the Walsh function actually does.
We see that we take the product of all of the terms of the vector $ a $ with some input vector $ b $.
If $ a_i = b_i = 1 $, then $ a_i b_i = 1 $; otherwise $ a_i b_i = 0 $.
We take each term $ a_i b_i $ and sum them to get some odd or even number, which results in either a $ -1 $ or $ 1 $ output.
In essence, the Walsh function issues a parity bit over the number of bits where $ a_i b_i = 1 $.
What we are curious about, though, is in what happens when we just change one bit of the input vector.
If $ a_i = 1 $, then changing the bit of the input vector will indeed change the result of the Walsh function.
If $ a_i = 0 $, then changing the bit of the input vector will not change the result of the Walsh function.
\\ \\
Now that we have some intuition of the Walsh function, let us inspect a particular round.
Suppose the host has picked vector $ b = (b_1, \ldots, b_6) $ and we are interested in divining the value of bit $ k $.
We take $ b_k $ and we flip it to produce $ c = (b_1, \ldots, \overline{b_k}, \ldots, b_6) $.
Let us investigate what happens to the result of the Walsh function.
\\ \\
Suppose $ b_k = 0 $.
If $ a_k = 0 $, then setting $ c_k = 1 $ would result in no change in the output of the Walsh function.
However, if $ a_k = 1 $, then setting $ c_k = 1 $ would result in a change in the parity of the Walsh function.
\\ \\
What happens when $ b_k = 1 $?
If $ a_k = 0 $, then setting $ c_k = 0 $ would result in no change in the output of the Walsh function.
However, if $ a_k = 1 $, then setting $ c_k = 0 $ would result in a change in the parity of the Walsh function.
\\ \\
What we have shown is that to divine the value of a particular $ a_k $, given some input $ b $, we simply need to flip bit $ k $.
Therefore, our strategy is as follows:
\begin{enumerate}
    \item For round $ k $, take the host's input $ b $ and flip bit $ k $ to produce $ c $.
    \item Ask if $ W_a(b) = W_a(c) $.
    \item If true, then $ a_k = 0 $; if false, then $ a_k = 1 $.
\end{enumerate}

Finally, we reach our \nth{7} round.
We will denote our final guess vector $ g $.
If there are no entries that are $ 1 $, i.e.\ we have determined that $ a = \vec{0} $, then ignore this round as it does not matter.
If there exists at least one entry that is a $ 1 $, pick the first entry.
Ask about its parity.
If it is $ + $, then we have the positive Walsh function characterized by $ a $, i.e.\ set $ g = a $.
If it is $ - $, then we have the negative Walsh function characterized by $ -a $, i.e.\ set $ g = -a $.
\\ \\
Now we can conclude: pick the door that corresponds to the Wash function characterized by $ g $, which will win us the Hawaiian vacation.
\QED{}

\section{Question 7}
\label{sec:Question7}

We will invoked a theorem that we proved on an earlier homework.
In particular, we know that the largest possible order of an element in a symmetric group $ S_N $ is the $ lcm $ of the partitions of $ N $.
For $ S_1 $, $ S_2 $, $ S_3 $, and $ S_4 $, this number is $ 1 $, $ 2 $, $ 3 $, and $ 4 $ respectively.
In $ S_5 $, note that there does exist an element of order $ 6 $, e.g.
 $ (123)(45) $.
\\ \\
Why is this important?
Simply, to construct a map $ \rho : G \to S_N $, we need to map $ G $ into a subgroup of $ S_N $.
However, to satisfy the requirement that $ \rho $ be injective and homomorphic, we must find an element in $ S_N $ of order $ 6 $.
Since there exists an element in $ C_6 $ of order $ 6 $ and $ \rho $ preserves group structure, this requirement must be met.
(Trivially, $ S_5 $ contains elements of all orders less than $ 6 $, so we know we have covered everything in $ C_6 $. This is seen by fixing $ i $ points, which results in a cycle of order of $ 6 - i $.)
By what we have shown above, $ N = 5 $ is the smallest such $ N $ for which this is possible.
Thus, by definition, $ d(C_6) = 5 $.
Additionally, $ |C_6| = 6 $.
Thus
\[
\alpha(C_6) = d(C_6) / |C_6| = 5/6
\]
\QED{}

\section{Question 8}
\label{sec:Question8}

\subsection{Part i}
\label{sub:8Parti}

First, note that we will always be able to find a faithful collection.
In particular, we see that the collection with just the trivial subgroup will indeed always be faithful.
Now, let us inspect the sums over faithful collections.
We are going to transform it a little by multiplying through with the constant $ |G| $, i.e.\ the order of the group:
\[
|G| \sum_{i}^{r} \frac{1}{|H_i|} = \sum_{i}^{r} \frac{|G|}{|H_i|}
\]

Multiplying through a sum with a non-zero constant, i.e. $ |G| $, still maintains this is a minimization problem.
Now, note that $ |A_i| = |G|/|H_i| $.
Thus, we are now trying to minimize the sums:
\[
\sum_i^{r} |A_i|
\]
Suppose that we have now indeed found such a minimal sum, which must exist since we must \textit{at least} one faithful collections.
Denote this minimal sum $ \sum_{i}^{r'} |A'_i| $
Since a faithful group action must induce an injective homomorphism into $ S_l $ for some $ l $, we can now draw from the definition of the Cayley Constant.
In essence, we see that we have to be able to sum the orders of the left cosets $ A'_i $, which will in fact correspond to partitions of $ S_N $.
For this to fit, we quite literally have to find an $ N $ large enough.
For example, we could map each $ A'_i $ to a cycle of length $ |A'_i| $.
The $ lcm $ of the partitions of $ S_N $ indicate how large we are able to get.
Therefore, drawing from our argument for Question~\ref{sec:Question7}, we see that $ d(G) $, i.e.\ the smallest such $ N $ for which there is an injective homomorphism will indeed correspond to the sums of the orders of $ A'_i $.
Therefore, we see that $ d(G) = \sum_i^{r} |A'_i| $.
Dividing by $ |G| $, we yield $ d(G) / |G| = \sum_i^{r} |A'_i| / |G| = \sum_i^{r} \frac{1}{|A'_i|} $.
Since $ d(G) / |G| = \alpha(G) $ (by definition), we can formally conclude:
\[
\alpha(G) = \min{\sum_i^{r} \frac{1}{|A_i|}}
\]

\subsection{Partii}
\label{sub:8Partii}

An interesting property of the quaternions is that all subgroups of the $ \mathbb{Q}_8 $ are normal.
In addition, if we inspect the lattice, we see that all subgroups ``go through'' the one generated by $ \langle -1 \rangle $.
\\ \\
Now, let us try to find faithful collections $ I $.
If we allow just one subgroup in this collection, we see that any subgroup besides the one generated by the identity ($ \langle 1 \rangle $) will result in a non-trivial normal subgroup in the intersection of the collection, namely itself.
In fact, if we try to find a faithful collection $ I $ without the subgroup generated by the identity, we will always have $ \langle -1 \rangle  $ in the intersection by the structure of this group.
Since $ \langle -1 \rangle $ is normal and is certainly nontrivial, we see that we must always include the $ \langle 1 \rangle $ in our colleciton $ I $ to generate a faithful collection.
\\ \\
We now reintroduce the result from the previous part.
Denote $ S $ as the subsgroup generated by $ \langle 1 \rangle $.
Note that $ |S| = 1 $.
Furthermore, $ 1 / |S| = 1 $.
In particular, note that this trivial subgroup is the only subgroup that has order $ 1 $.
Therefore, it also is the only subgroup with inverse order $ 1 $.
Thus, any sum over the inverse orders of the subgroups that is a faithful collection for the quaternions will have to be \textit{at least} greater than or equal to $ 1 $.
For all faithful collecitons $ I_i $, designate this sum of inverse orders as $ \sigma(I_i) = \sigma_i $
In short, we know that $ \sigma_i \geq 1 $
\\ \\
Finally, pick the collection $ I_0 = \{S\} $.
This clearly only has the trivial subgroup as its intersection.
Now, we can clearly see that $ \sigma_0 = 1 $.
Thus to formally conclude, we see that since $ \sigma_i \geq 1 $ and that we have indeed found a faithful collection $ I $ where the equality is true, i.e.\ $ \sigma_0 $, that the minimum across the $ \sigma_i $s is in fact $ 1 $.
Therefore, $ \alpha\left(\mathbb{Q}_8\right) = 1 $.
\QED{}

\end{document}
