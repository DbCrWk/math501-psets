\documentclass[letterpaper]{article}

\usepackage{amsmath}
\usepackage{amssymb}
\usepackage{hyperref}
\usepackage{geometry}
\usepackage{nth}

% Comment the following lines to use the default Computer Modern font
% instead of the Palatino font provided by the mathpazo package.
% Remove the 'osf' bit if you don't like the old style figures.
\usepackage[T1]{fontenc}
\usepackage[sc,osf]{mathpazo}

\newcommand*{\QED}{\hfill\ensuremath{\square}}%

% Set your name here
\def\name{Problem Set 4}


\geometry{
  body={6.5in, 8.5in},
  left=1.0in,
  top=1.25in
}

% Customize page headers
\pagestyle{myheadings}
\markright{\name}
\thispagestyle{empty}

% Custom section fonts
\usepackage{sectsty}
\sectionfont{\rmfamily\mdseries\Large}
\subsectionfont{\rmfamily\mdseries\itshape\large}

% Other possible font commands include:
% \ttfamily for teletype,
% \sffamily for sans serif,
% \bfseries for bold,
% \scshape for small caps,
% \normalsize, \large, \Large, \LARGE sizes.

% Don't indent paragraphs.
\setlength\parindent{0em}

\begin{document}

{\huge \name}

% Alternatively, print name centered and bold:
%\centerline{\huge \bf \name}

\vspace{0.25in}

Dev Dabke \\
MATH 501: Introduction to Algebraic Structures I \\
September 27, 2016 \\
Prof.\ Calderbank \\

\section{Question 1}
\label{sec:Question1}
Without loss of generality, we will consider just the fundamental tiles centered at the origin.
\\ \\
To provide some intuition, we are trying to find, essentially, the areas of the different shapes produced in this method.
First, we produce a hexagon that represents a Voronoi region for the generator matrix.
We also produce a skewed rectangle (i.e.\ a parallelogram) that represents our transformation applied to the Voronoi region of a simple squre.
In particular, we want to see, scaled to a origin-centered unit, how much of the parallelogram fits in the hexagon.
\\ \\
Note that in particular, the Voronoi region for the lattice produced by the standard basis vectors is a square.
Any point whose $ x $ coordinate and $ y  $ coordinate less than $ 1/2 $ would be rounded to the origin.
Therefore, we end up tracking $ (1/2, 1/2) $, $ (-1/2, 1/2) $, $ (1/2, -1/2) $, and $ (-1/2, -1/2) $ through this transformation.
We see that we create a parallelogram between the points $ (3/4, \sqrt{3}/4) $, $ (1/4, -\sqrt{3}/4) $, $ (-1/4, \sqrt{3}/4) $, and $ (-3/4, -\sqrt{3}/4) $.
\\ \\
Now, the Voronoi region of the basis represented by the generator matrix is a hexagon with vertices at $ (\pm 1/4, \sqrt{3}/4) $, $ (\pm 1/2, 0) $, and $ (\pm 1/4, -\sqrt{3}/4) $.
Thus, we see that geometrically, we leave out the two equilateral triangles that are part of the parallelogram and not part of the hexagon.
These triangles represent possible values of $ (e_1, e_2) $ that give rise to an $ (m_1, m_2) $ that become an $ (l_1, l_2) $ that lie outside of the Voronoi region of fundamental tile of the generator matrix.
In short, these points $ (x_1, x_2 $ will not produced $ (l_1, l_2) $ that are the closest in the lattice.
\\ \\
To perform the computation, see that the area of the hexagon is $ 6 $ triangles, whereas the area of the parallelogram is that plus $ 2 $ more triangles.
Thus, the area of the parallelogram is $ 8 $ triangles.
There is a $ \frac{6}{8} $ chance that our algorithm will work.
\\ \\
See attached proof without words for clarification and the bonus.
\QED{}


\section{Question 2}
\label{sec:Question2}

\subsection{Part i}
\label{sub:2Parti}

For are trying to prove that $ \Lambda $ is indeed a sublattice of $ \Lambda / 2 $.
Fortunately, based on the statement of the question, we can take for granted that $ \Lambda / 2 $ is indeed a lattice.
In addition, since $ \Lambda $ is also a lattice by assumption, we know that it already satisfies all of the requirements for its operators.
Thus, we need only prove that $ \Lambda $ is a subset of $ \Lambda / 2 $.
\\ \\
We want to show that we have all elements of $ \Lambda $ in $ \Lambda / 2 $.
This is quite easy to prove, since we can simply take every integer combination in $ \Lambda $ and multiply through by $ 2 $.
This is trivially shown by
$$ \sum_{i = 1}^{8} d_i v_i = \sum_{i = 1}^{8} 2 d_i \frac{1}{2} v_i = \sum_{i = 1}^{8} 2 d_i w_i $$
To make this rigorously correct, we should point out that any integer multiplied by $ 2 $ is still an integer.
This fact follows from the the fact that the integers form a multiplicative group.
Thus, we can conclude that $ \Lambda $ is a subset of $ \Lambda $.
\\ \\
First, let us produce $ M / 2 $, which is simply $ M $ with each entry divided by $ 2 $.
Note that each new vector $ v_i / 2 $ is a scalar multiple of $ v_i $, i.e.\ it lies in its span.
Because of this fact, we know that these new vectors must also be linearly independent.
\\ \\
Now, we will produce $ M / 2 $, which we can easily show is a valid generator matrix for $ \Lambda / 2 $.
In essence, if we take a scalar multiple of a basis vector, we still have produced a valid basis vector.
To prove this more rigorously, note that, since the $ v_i $s are linearly independent, $ \sum_{i = 1}^{8} c_i v_i = 0 \implies \forall i, \, c_i = 0 $.
Now, let $ w_i = \frac{1}{2} v_i $ be the new vectors of $ M / 2 $.
$ \sum_{i = 1}^{8} c_i w_i = 0 \iff \sum_{i = 1}^{8} c_i \frac{1}{2} v_i = 0 \implies \frac{1}{2} \sum_{i = 1}^{8} c_i v_i = 0 \implies \sum_{i = 1}^{8} c_i v_i = 0 $.
Since $ \sum_{i = 1}^{8} c_i w_i = 0 \implies \sum_{i = 1}^{8} c_i v_i = 0 $ and $ \sum_{i = 1}^{8} c_i v_i = 0 \implies c_i = 0 $, we know that $ \sum_{i = 1}^{8} c_i w_i = 0 \implies c_i = 0 $.
\\ \\
Thus, now that we have a generator matrix for each lattice, we can compute their fundamental volumes.
To compute easily, we note that the determinant of an upper triangular matrix is the product of its diagonals.
Furthermore, the determinant is a linear operation, and so $ \det{(A A^T)} = \det{(A)} \det{(A^T)} $.
This math we can do in our head, and so we get that the fundamental volume of $ \Lambda $ using $ M $ is $ 2^8 $ and the fundamental volume of $ \Lambda / 2 $ using $ M / 2 $ is $ 1 $.
(n.b.\ this is even nicer since the transpose of an upper triangular matrix is lower triangular. The determinant of a lower triangular matrix is also its diagonal, which remains the same under transposition of a square matrix. Furthermore, we are only dealing with powers of two, which is quite easy to operate on.)
\\ \\
Finally, we can easily see the index (which by definition is the ratio of the fundamental volumes) is $ 2^8 $.
\QED{}

\subsection{Part ii}
\label{sub:2Partii}

Note that a particular $ z \in \Lambda $ can be written as a linear combination of the $ v_i $s.
The inner product of $ z $ with itself (i.e.\ $ (z, z) $) gives us the $ ||z||^2 $ that we looking for, in that $ (z, z) = ||z||^2 $.
Since the inner product is the sum of the pairwise products of all of the terms of $ z $, we obtain a formula as follows:
$$ (z, z) = \sum_{i, j} \lambda_i \lambda_j (v_i, v_j) $$
Now, for every pair $ (i, j) $, we add both $ \lambda_i \lambda_j (v_i, v_j) $ and $ \lambda_j \lambda_i (v_j, v_i) $.
Since $ (v_i, v_j) = (v_j, v_i) $ and $ \lambda_i \lambda_j = \lambda_j \lambda_i $, we know that $ \lambda_i \lambda_j (v_i, v_j) = \lambda_j \lambda_i (v_j, v_i) $.
Therefore, we are adding $ 2 \lambda_i \lambda_j (v_i, v_j) $.
\\ \\
Now, we will consider two cases:
for $ i \neq j $, we know that $ (v_i, v_j) \equiv 0 \mod{4} $.
A number that is $ 0 \mod{4} $ must either be $ 0 \mod{8} $ or $ 4 \mod{8} $.
However, since $ 2 $ is trivially $ 2 \mod{8} $, we see that $ (0)(2) = 0 \mod{8} $ and $ (4)(2) = 0 \mod{8} $.
Therefore, $ 2 (v_i, v_j) \equiv 0 \mod{8} $.
\\ \\
Next, let us consider the case $ i = j $.
In this case, we simply look to the matrix and notice that $ (v_i, v_i) \equiv 8 \mod{8} $ by a brute force computation.
\\ \\
At this point, we have shown that $ 2 (v_i, v_j) \equiv 0 \mod{8} $ for any $ i $ or $ j $, regardless of whether or not they are the same.
Now, by the associativity of multiplication of integers, note that $ 2 \lambda_i \lambda_j (v_i, v_j) = \lambda_i \lambda_j 2 (v_i, v_j) $.
Since $ 2 (v_i, v_j) \equiv 0 \mod{8} $, we see that $ \lambda_i \lambda_j 2 (v_i, v_j) \equiv \lambda_i \lambda_j (0) \mod{8} $.
This directly implies that $ 2 \lambda_i \lambda_j (v_i, v_j) \equiv \lambda_i \lambda_j (0) \mod{8} $.
Since this is a residue class, any integer multiple of $ 0 \mod{8} $ is still $ 0 \mod{8} $.
Therefore, $ \lambda_i \lambda_j (0) \equiv 0 \mod{8} $.
Since for every term in the summation $ \lambda_i \lambda_j (v_i, v_j) $ we also add $ \lambda_j \lambda_i (v_j, v_i) $ which means we are in fact adding $ 2 \lambda_i \lambda_j (v_i, v_j) $.
Since $ 2 \lambda_i \lambda_j (v_i, v_j) \equiv 0 \mod{8} $, we see that we can transform our summation of the $ \lambda_i \lambda_j (v_i, v_j) $ terms into a summation of elements that are $ 0 \mod{8} $.
\\ \\
Returning back to our original few facts, we have shown that $ ||z^2|| $ is the sum of elements who are $ 0 \mod{8} $.
Since an additive residue class has as its identity $ 0 $, the sum of all of these values also has to be $ 0 $.
We can formally conclude that $ z \in \Lambda \implies ||z||^2 \equiv 0 \mod{8} $.
\QED{}

\subsection{Part iii}
\label{sub:2Partiii}

We see that $ 8 $ must divide into $ ||z||^2 $ by the previous part.
Take any element $ z \in \Lambda $ and take its counterpart in $ \Lambda / 2 $, i.e.\ $ \frac{z}{2} $.
We see that $ ||\frac{z}{2}||^2 = \frac{1}{4} ||z||^2 $.
Thus, we see that $ 2 $ must divide $ ||\frac{z}{2}||^2 = d^2 (\Lambda / 2) $.
Therefore, the minimum squared distance, by definition the smallest squared distance greater than $ 0 $, must be $ 2 $, since $ 2 $ is the smallest integer that $ 2 $ divides that is not $ 0 $.
\QED{}

\section{Question 3}
\label{sec:Question3}

\subsection{Part i}
\label{sub:3Parti}

Figure attached.
Note that for $ \Lambda $, we have generator matrix $ M = \begin{bmatrix} 1 & 0 \\ 0 & \sqrt{5} i \end{bmatrix} $.
$ V(\Lambda) = \sqrt{5} $.
Now, if multiply $ (1 + \sqrt{5} i) $ against a generic number in the form $ (a + \sqrt{5} i b) $, we yield:
$$ (1 + \sqrt{5} i)(a + \sqrt{5} i b) = a + \sqrt{5} i a + \sqrt{5} i b - 5b = (a - 5b) + (a + b) \sqrt{5} i $$
Thus for $ (1 + \sqrt{5} i)\Lambda $ we have generator matrix $ N = \begin{bmatrix} 1 & -5 \\ \sqrt{5} i & \sqrt{5} i \end{bmatrix} $.
Computing with some computational help: $ V((1 + \sqrt{5} i)\Lambda) = 6 \sqrt{5} $.
As a sanity check, note that $ (1 + \sqrt{5} i) (1 - \sqrt{5} i) = 6 $, so satisfyingly, we see that $ (1 + \sqrt{5} i) (1 - \sqrt{5} i) V(\Lambda) = V((1 + \sqrt{5} i)\Lambda) $.
This sanity check essentially describes the quadratic relationship between a scaled rotation in $ \mathbb{R}^2 $ and the fundamental volume (i.e.\ area in this case).
\QED{}

\subsection{Part ii}
\label{sub:3Partii}

Geometrically, we multiplying by $ (1 + \sqrt{5} i) $ is a scaled rotation.
In particular, we are scaling by
\[
\left\lVert (1, \sqrt{5}) \right\rVert = \sqrt{6}
\]
and rotating by
\[
\theta = \tan^{-1}{\left(\frac{\sqrt{5}}{1}\right)}
\]

In general, multiplying by $ (a + b \sqrt{5} i) $ is a scaled rotation, with scaling:
\[
\left\lVert (a, b \sqrt{5}) \right\rVert = \sqrt{a^2 + 5 b^2}
\]
and rotating by
\[
\theta = \tan^{-1}{\left(\frac{b \sqrt{5}}{a}\right)}
\]
We can see this intuition from either the Pythagorean identities, which essentially treats each component of the basis vector as one leg of a triangle.
We can also convince ourselves through Euler's identity.
Essentially, the magnitude of the vector represents the \textit{scale} factor, or how much we are stretching or shrinking the basis.
The ratio of the terms of the vector represents how much we are rotating the basis.
\QED{}

\subsection{Part iii}
\label{sub:3Partiii}

See attached.

\subsection{Part iv}
\label{sub:3Partiv}

Without loss of generality, note that:
\[
(a + b\sqrt{5} i) \Lambda = (c + d\sqrt{5} i) \Lambda'
\]
can be rewritten as
\[
\left[(a + b\sqrt{5} i) - (c + d\sqrt{5} i) \right] \Lambda = \Lambda'
\]
which can be further reduced to
\[
\left[(a - c) + (b - d) \sqrt{5} i \right] \Lambda = \Lambda'
\]
Since $ a $, $ b $, $ c $, and $ d $ were picked as generic integers, and indeed can be added, subtracted, and multiplied to make other integers, we can allow $ k = a - c $ and $ l = b - d $ without loss of generality.
Thus, we seek:
\[
(k + l \sqrt{5} i) \Lambda = \Lambda'
\]
In essence, we are trying to find a scaled rotation that transforms $ \Lambda $ into $ \Lambda' $.
This is not possible.
The basis vectors $ (1, 0) $ and $ (0, \sqrt{5}) $ cannot be transformed into the basis vectors $ (2, 0) $ and $ (1, \sqrt{5}) $ with integer coefficients.
Perhaps even more persuasively is our drawing.
We see that $ \Lambda $ has as a fundamental tile a rectangle, whereas we have a hexagon for the fundamental tile of $ \Lambda' $.
There is simply no way to transform one into the other with a simple scaled rotation.
\QED{}

\end{document}
