\documentclass[letterpaper]{article}

\usepackage{amsmath}
\usepackage{amssymb}
\usepackage{hyperref}
\usepackage{geometry}
\usepackage{nth}

% Comment the following lines to use the default Computer Modern font
% instead of the Palatino font provided by the mathpazo package.
% Remove the 'osf' bit if you don't like the old style figures.
\usepackage[T1]{fontenc}
\usepackage[sc,osf]{mathpazo}

% Set your name here
\def\name{Problem Set 1}


\geometry{
  body={6.5in, 8.5in},
  left=1.0in,
  top=1.25in
}

% Customize page headers
\pagestyle{myheadings}
\markright{\name}
\thispagestyle{empty}

% Custom section fonts
\usepackage{sectsty}
\sectionfont{\rmfamily\mdseries\Large}
\subsectionfont{\rmfamily\mdseries\itshape\large}

% Other possible font commands include:
% \ttfamily for teletype,
% \sffamily for sans serif,
% \bfseries for bold,
% \scshape for small caps,
% \normalsize, \large, \Large, \LARGE sizes.

% Don't indent paragraphs.
\setlength\parindent{0em}

\begin{document}

{\huge \name}

% Alternatively, print name centered and bold:
%\centerline{\huge \bf \name}

\vspace{0.25in}

Dev Dabke \\
MATH 501: Introduction to Algebraic Structures I \\
September 6, 2016 \\
Prof.\ Calderbank \\

\section{Problem 1}
\label{sec:Problem1}

Using Euclid's algorithm, we see
\begin{align}
    1761 & = (1) 1567 & + 194 \\
    1567 & = (8) 194 & + 15 \\
    194 & = (12) 15 & + 14 \\
    15 & = (1) 14 & + 1 \\
    14 & = (14) 1 & + 0
\end{align}

Thus, the $ \gcd(1761, 1567) = 1 $ (this makes sense since $ 1567 $ is prime).
Furthermore, if we ``bubble'' up the coefficients, we see that:
$$ (-105) 1761 + (118) 1567 = 1 = \gcd(1761, 1567) $$

\section{Problem 2}
\label{sec:Problem2}

We will prove this result by induction using the following hypothesis:
$$ r_n = s_n a + t_n b $$

First, to prove the base case $ n = 1 $, we see that $ s_1 = s_{-1} - q_1 s_0 = 1 - 0 = 1 $ and that $ t_1 = t_{-1} - q_1 t_{0} = 0 - q_1(1) = -q_1 $.
Moreover, note that by definition of Euclid's algorithm (with some rearranging), $ r_1 = a - q_1 b = r_1 = s_1 a + t_1 b $.
\\ \\
Now, assuming that our inductive hypothesis holds, we move to calculate $ r_{n + 1} $. By Euclid's algorithm, we see that:
$$ r_{n - 1} = q_n r_n + r_{n + 1} $$

Substituting for $ r_{n - 1} $ and $ r_n $, we yield:
$$ s_{n - 1} a + t_{n - 1}b = q_n (s_n a + t_n b) + r_{n + 1} $$

Rearranging:
$$ s_{n - 1}a + t_{n - 1}b - q_n s_n a - q_n t_n b = r_{n + 1} \implies (s_{n - 1} - q_n s_n)a + (t_{n - 1} - q_n t_n)b = r_{n + 1} $$

By definition of $ s_{n + 1} $ and $ t_{n + 1} $, we see that we have proved our result:
$$ s_{n + 1}a + t_{n + 1}b = r_{n } 1 $$

Finally, since this holds for all $ n \geq 1 $, by definition of Euclid's algorithm, we know that $ r_{ j + 1} = 0 $ implies $ r_j = \gcd(a, b) $. Thus, for such $ r_j $, $ r_j = s_j a + t_j b = \gcd(a, b) $.

\section{Problem 3}
\label{sec:Problem3}

We will also prove this result by induction with two hypotheses:
$$ \gcd(F_n, F_{n - 1}) = 1 $$ and that $ F_n $ is odd.
\\ \\
To begin, we see that $ F_2 = 1 $. Thus, $ \gcd(F_2, F_1) = \gcd(1, 1) = 1 $, which satisfies our hypotheses.
To prove that $ F_{n + 1} $ is odd, we quickly see that we are adding an odd number, $ F_n $, to an even number $ 2F_{n - 1} $.
\\ \\
Now that we have proven that $ F_n $ is always odd, we note that in this interesting, Fibonacci-esque sequence, it is evident by our inductive hypothesis $ \gcd(F_n, F_{n - 1}) = 1 $ that $ \gcd(F_n, 2F_{n - 1}) = 1 $ must also be true (i.e.\ $ 2 $ will never be a factor of $ F_n $).
By the relation $ \gcd(a + b, b) = \gcd(a, b) $, $ \gcd(2F_{n - 1}, F_n) = \gcd(2F_{n - 1} + F_n, F_n) = 1 $.
Substituting $ 2F_{n - 1} + F_n = F_n + 2F_{n - 1} = F_{n + 1} $, we arrive at our conclusion:
$$ \gcd(F_{n + 1}, F_n) = 1 $$

\section{Problem 4}
\label{sec:Problem4}

$$ 9^{1500} \equiv 01 \mod 100 $$

We arrive at the above conclusion by using some rules of residue classes (n.b.\ this problem is always $ \mod 100 $).
Precisely, note that $ 9^{5^{5^{5^{3^{2^2}}}}} = 9^{1500} $.
As horrible as this looks, we actually have a very elegant result in that $ 9^5 \equiv 49 $ and $ 49^5 \equiv 01 $.
Thus, $ 9^{5^{5^{5^{3^{2^2}}}}} \equiv 49^{5^{5^{3^{2^2}}}} \equiv 01^{5^{3^{2^2}}} \equiv 01 $.

\section{Problem 5}
\label{sec:Problem5}

\subsection{Part i}
\label{subs:Parti}

This statement is false.
\\ \\
With the vector space $ \mathbb{R}^2 $ and vector addition $ + $ as our group operator, we note that we can define two subgroups $ S_0 = span\left(\begin{bmatrix} 1 \\ 0 \end{bmatrix}\right) $ and $ S_1 = span\left(\begin{bmatrix} 0 \\ 1 \end{bmatrix}\right) $.
These subgroups inherit all of the properties of the group operator (e.g.\ associativity, well-defined inverses, etc.) and are closed under the group operator by definition of being spans of vectors.
Note that $ \begin{bmatrix} 1 \\ 1 \end{bmatrix} $ is neither in $ S_0 $ nor in $ S_0 $, and thus must be excluded from $ S_0 \cup S_1 $. However, $ \begin{bmatrix} 1 \\ 0 \end{bmatrix} $ and $ \begin{bmatrix} 0 \\ 1 \end{bmatrix} $ are in this union (by virtue of being in their respective sets).
Thus, this union is not closed under the group operator, since $ \begin{bmatrix} 1 \\ 0 \end{bmatrix} + \begin{bmatrix} 0 \\ 1 \end{bmatrix} = \begin{bmatrix} 1 \\ 1 \end{bmatrix} $.

\subsection{Part ii}
\label{subs:Partii}

This statement is true.
\\ \\
Given subgroups $ S_0 $ and $ S_1 $, note that their intersection inherits all of the properties of a given group operator (e.g.\ associativity, well-defined inverses, etc.).
The only question is: will the intersection be closed under this operator?
Yes.
Very briefly, we see that for $ a \in S_0 \cap S_1 $ and $ b \in S_0 \cap S_1 $, which implies $ a, b \in S_0, S_1 $.
This implies that $ (a, b) \in S_0 $ (by definition of a subgroup) and $ (a, b) \in S_1 $ (symmetrically).
Thus, $ (a, b) \in S_0 \cap S_1 $, demonstrating that the intersection of two subgroups is indeed a subgroup.

\subsection{Part iii}
\label{subs:Partiii}

This statement is false.
\\ \\
For the cyclic group $ C_6 $ generated by $ \langle x | x^6 = 1 \rangle $, we see that $ xx^3 = x^4 $.
$ |x^4| = 2 $, but $ |x| = 6 $ and $ |x^3| = 3 $, so $ lcm(6, 3) = 6 \neq 2 $.

\subsection{Part iv}
\label{subs:Partiv}

This statement is false.
\\ \\
We will use the group of nonsingular 2-by-2 matrices with our group operator as matrix addition.
Take $ A = \begin{bmatrix} 1 & 2 \\ 1 & 0 \end{bmatrix} $ with inverse $ A^{-1} = \frac{1}{2} \begin{bmatrix} 0 & 2 \\ 1 & -1 \end{bmatrix} $ and $ B = \begin{bmatrix} 1 & 3 \\ 1 & 0 \end{bmatrix} $ with inverse = $ B^{-1} = \frac{1}{3} \begin{bmatrix} 0 & 3 \\ 1 & -1 \end{bmatrix} $.
$ {(AB)}^{-1} = \frac{1}{6} \begin{bmatrix} 3 & -3 \\ -1 & 3 \end{bmatrix} $ whereas $ A^{-1}B^{-1} = \frac{1}{6} \begin{bmatrix} 2 & -2 \\ -1 & 4 \end{bmatrix} $.

\subsection{Part v}
\label{subs:Partv}

This statement is true.
\\ \\
The map $ \phi : A \mapsto {(A^T)}^{-1} $ preserves group structure, in that
$$ \phi(AB) = \left[{(AB)}^T\right]^{-1} = {(B^T A^T)}^{-1} = A^{T^{-1}} B^{T^{-1}} = \phi(A) \phi(B) $$

\section{Problem 6}
\label{Problem6}

Using Euclid's algorithm (work done on attached scratch paper, but only the results are reproduced here):

\begin{align}
    x^8 & = (x^2 + 1) (x^6 + x^4 + x^2 + x + 1) & + (x^3 + x + 1) \\
    (x^6 + x^4 + x^2 + x + 1) & = (x^3 + 1) (x^3 + x + 1) & + x^2 \\
    (x^3 + 1) & = (x) (x^2) & + (x + 1) \\
    x^2 & = (x + 1) (x + 1) & + 1 \\
    (x + 1) & = (1) (x + 1) & + 0
\end{align}

Thus, $ \gcd(x^8, x^6 + x^4 + x^2 + x + 1) = 1 $.
\\ \\
Furthermore:
$$ s = (x^5 + x^4 + x^3 + x^2), t = x^7 + x^6 + x^3 + x + 1 $$
where $ \gcd(a, b) = sa + tb $ and $ a = x^8, b = x^6 + x^4 + x^2 + x + 1 $.

\section{Problem 7}
\label{Problem7}

\subsection{Part i}
\label{Parti}
This transformation is intuitively represented by $ zxz $ which reduces to $ -x $.

\subsection{Part ii}
\label{Partii}
This transformation is intuitively represented by $ xzx $ which reduces to $ -z $.

\subsection{Part iii}
\label{Partiii}
This transformation could be thought of as a counter-clockwise rotation of $ \frac{\pi}{4} $ and then a reflection across the x-axis.

\subsection{Part iv}
\label{Partiv}
The elements in $ D_4 $ with their respective permutations are:

\begin{align}
1 & \; & (A_1)(A_2)(B_1)(B_2) \\
y = xz & \; & (A_1 A_2)(B_1)(B_2) \\
y^2 & \; & (A_1)(A_2)(B_1)(B_2) \\
y^3 &\; & (A_1 A_2)(B_1B_2) \\
z &\; & (A_1 A_2)(B_1B_2) \\
yz &\; & (A_1 A_2)(B_1B_2) \\
y^2 z & \; & (A_1)(A_2)(B_1B_2) \\
y^3 z & \; & (A_1A_2)(B_1)(B_2)
\end{align}

\section{Problem 8}
\label{Problem8}

Note that every element can be expressed as either $ y^i $ or $ y^i z $ (see the textbook).
For any element in the form $ y^i z $, we see that $ \left|y^i z\right|^2 = (y^i z)(y^i z) = y^i z y^i z $.
Now by the second relation, we can ``move'' the right-most $ z $ into the middle.
Thus $ \left|y^i z\right|^2 = y^i z z y^{-i} = y^i y^{-i} = 1 $.
We can conclude that any element of the form $ y^i z $ has order $ 2 $.
\\ \\
Furthermore, by the given relation $ z $ has order $ 2 $.
Similarly, since $ yz $ is in the form expressed above, it also has order $ 2 $.
Thus, since $ z = z $ and $ yzz = y $, we can see that the original generators $ y, z $ can be transparently transformed into $ z $ (identity) and $ yz $.
We can then conclude that $ D_N $ is generated by $ z, yz $.

\section{Problem 9}
\label{Problem9}

For an arbitrary element in the form $ y^i $, note that it commutes with any other element in the form of $ y^j $, i.e. $ y^i y^j = y^j y^i $ by associativity.
Now we will see what happens when we attempt to commute with terms in the form $ y^j z $.
$ y^i y^j z = y^j z y^{-i} $ by associativity and the second relation.
\\ \\
Multiplying $ y^n = 1 $ on the right (which does not change the value), we get $ y^j z y^{-i} y^N = y^j z y^{2k - i} $ which only equals $ y^i y^j z $ if and only if $ k = i $.
\\ \\
Note that if you start with an element in the form $ y^i z $, then the proof still holds, since an even number of $ z $s will ``disappear'' (i.e.\ when against $ y^j z $) by the first relation and an odd number (i.e.\ when against $ y^j $) will produce the same result.

\end{document}
