\documentclass[letterpaper]{article}

\usepackage{amsmath}
\usepackage{amssymb}
\usepackage{hyperref}
\usepackage{geometry}
\usepackage{nth}

% Comment the following lines to use the default Computer Modern font
% instead of the Palatino font provided by the mathpazo package.
% Remove the 'osf' bit if you don't like the old style figures.
\usepackage[T1]{fontenc}
\usepackage[sc,osf]{mathpazo}

\newcommand*{\QED}{\hfill\ensuremath{\square}}%

\makeatletter
\renewcommand{\@seccntformat}[1]{%
  \ifcsname prefix@#1\endcsname
    \csname prefix@#1\endcsname
  \else
    \csname the#1\endcsname\quad
  \fi}
% define \prefix@section
\newcommand\prefix@section{Question \thesection}
\makeatother

\DeclareMathOperator{\rank}{rank}
\DeclareMathOperator{\lcm}{lcm}
\DeclareMathOperator{\tr}{Tr}

% Set your name here
\def\name{Problem Set 10}


\geometry{
  body={6.5in, 8.5in},
  left=1.0in,
  top=1.25in
}

% Customize page headers
\pagestyle{myheadings}
\markright{\name}
\thispagestyle{empty}

% Custom section fonts
\usepackage{sectsty}
\sectionfont{\rmfamily\mdseries\Large}
\subsectionfont{\rmfamily\mdseries\itshape\large}

% Other possible font commands include:
% \ttfamily for teletype,
% \sffamily for sans serif,
% \bfseries for bold,
% \scshape for small caps,
% \normalsize, \large, \Large, \LARGE sizes.

% Don't indent paragraphs.
\setlength\parindent{0em}

\begin{document}

{\huge \name}

% Alternatively, print name centered and bold:
%\centerline{\huge \bf \name}

\vspace{0.25in}

Dev Dabke \\
MATH 501: Introduction to Algebraic Structures I \\
November 22, 2016 \\
Prof.\ Calderbank \\

\section{}
\label{sec:Question1}

These were solved with some brute force computations.

\subsection{Part i}
\label{subsec:1Parti}

\[
\begin{bmatrix}
    1 & 0 \\
    0 & 7
\end{bmatrix}
\]
\QED{}

\subsection{Part ii}
\label{subsec:1Partii}

\[
\begin{bmatrix}
    1 & 0 & 0 \\
    0 & 3 & 0
\end{bmatrix}
\]
\QED{}

\subsection{Part iii}
\label{subsec:1Partiii}

\[
\begin{bmatrix}
    1 & 0 & 0 \\
    0 & 1 & 0 \\
    0 & 0 & 44
\end{bmatrix}
\]
\QED{}


\section{}
\label{sec:Question2}

Note that $ g^t = \begin{bmatrix} 1 & t \\ 0 & 1 \end{bmatrix} $.
In addition, $ \det{g} = \det{g^t} = \det{h} = 1 $.
Thus, $ g^t $ represents adding $ t $ times the second row to the first.
Apply $ h $ is a row switch and negating one row.
Since we can also apply $ g^{-t} $, we can arbitrary apply $ t $ times the first row to the second if perform $ hg^{-t} $.
Therefore, we can arbitrarily add one row to the other.
\\ \\
Next, we see that $ h^2 = -I_2 $, $ h^3 = -h $, and $ h = I_2 $ (like an imaginary unit!).
This allows us arbitrary row swaps (there is only one row swap when $ n = 2 $).
However, we have the additional restriction that elements of $ SL_2(\mathbb{Z}) $ have determinant $ 1 $.
Since row swapping would change the parity of a matrix's determinant, we must therefore only perform ``corrective'' row swaps (i.e.\ correct for the determinant).
\\ \\
Using our linear knowledge, we know that all nonsingular matrices can be generated via the three types of elementary matrices.
Since elements of $ SL_2(\mathbb{Z}) $ have determinant $ 1 $, they are all nonsingular, so we will investigate if we have properly captured all elementary matrices with $ g $ and $ h $.
Since $ \det{AB} = \det{A}\det{B} $, i.e.\ the determinant of the product of matrices is the product of their determinants, we know that the determinant of arbitrary strings of $ g $ and $ h $ will still be $ 1 $.
\\ \\
The first class of elementary matrices represents adding arbitrary amounts of one row to another; by what we have shown above, this is possible with $ g $ and $ h $.
Second, we can perform row swaps, but by the additional determinant means that we have to perform row swaps and negate a row to maintain the determinant.
Since we perform the operation on strings of $ g $ and $ h $, we know that we since we start with a matrix of determinant $ 1 $, we can yield all possible ``legal'' row swaps that preserve the determinant.
Finally, the last class of elementary matrices is multiplying a row by a scalar, but this class of elementary matrices also scales the determinant.
Therefore, the only ``legal'' elementary matrix of this type is $ I_2 $, which we know we can generate as $ h^4 $.
\\ \\
Thus, given strings of $ h $ and $ g $, we can generate all determinant-preserving elementary matrices.
We can start with $ h^4 = I_2 $, whose determinant is $ 1 $, and generate any other nonsingular matrix with determinant $ 1 $ with strings of $ g $ and $ h $.
Thus we can formally conclude by saying that $ g $ and $ h $ generate $ SL_2(\mathbb{Z}) $.
\QED{}

\section{}
\label{sec:Question3}

Note that all field elements that are not $ 0 $ and $ 1 $ have inverses.
Plus, $ xx^{-1} = 1 $.
In addition, $ x0 = 0 $, and in particular $ 0 \times 1 = 0 $.
Therefore, allowing $ \Omega = \{0, \, 1\} $ satisfies the two properties of the complete set of non-associates: first, every other element in the ring (field) associate with some element in $ \Omega $ and the elements of $ \Omega $ do not associate (since $ 0 \times 1 = 0 $).
Thus, $ \{0, \, 1\} $ is the set of non-associates for any field.
\\ \\
We now apply the Hermite Normal Form Theorem (Theorem 21.2, Lecture 21, Slide 3).
In general, we know that any $ N \times N $, possibly singular matrix is the left associate of an $ N \times N $ matrix with zeros above the main diagonal.
We know that each diagonal element lies in a prescribed system of non-associates, which in this case implies that each diagonal element is either $ 1 $ or $ 0 $.
Furthermore, if the diagonal element $ a \neq 0 $, then each element below the main diagonal lies in a complete residue system modulo $ a $, but since the only $ a : \neq = 0 $ is $ a = 1 $, and everything mod $ 1 $ is $ 0 $, we know that all entries below (and above) this diagonal entry is $ 0 $.
Therefore, our matrix has columns that take one of two forms: a $ 1 $ on the entry that falls on the main diagonal, and $ 0 $s everywhere else in the column; or a $ 0 $ on the entry that falls on the main diagonal and possibly other values below it.
\QED{}

\section{}
\label{sec:Question4}
Pick $ U $ such that $ UA $ is in Smith Normal Form.
We know that $ UA $ is a diagonal matrix.
By Question~\ref{sec:Question3}, we know that if $ A $ is square and is possibly singular, and $ A $ has elements in a field $ F $, then its determinental divisors have to come from a prescribed set of non-associates.
In particular, this set of non-associates is $ \{0, \, 1\} $.
Since the entries of $ UA $ are from this prescribed set of non-associates, we know that therefore $ UA $ is a diagonal matrix with entries $ \{0, \, 1\}$.
Note that for any diagonal matrix squared, the resultant square is also a diagonal matrix, with each entry being the square of the entry in that same position.
Since $ 0^2 = 0 $ and $ 1^1 = 1 $, we know that $ {(UA)}^2 = UA $ since all of its elements will square to themselves.
\QED{}

\section{}
\label{sec:Question5}

Take $ H' $ as $ H $ is Smith Normal Form.
We know that $ \det{H} = \det{H'} $.
Thus, we see that $ \det{HH^T} = \det{H'H'^T} = \det{N I_N} = N^N $.
In addition, $ \det{H'H'^T} = h_1^2 \cdots h_r^2 $, where the $ h_i $s are the invariants.
Thus, $ h_1^2 \cdots h_r^2 = N^N $.
\\ \\
By Theorem 21.7, Lecture 21, Slide 11, we know that these matrices, $ HH^{T} $ and $ N I_N $ have the same determinental divisors since they are equivalent.
The determinental divisors  $ d_i $ are obvious for $ N I_N $ and are in general of the form $ d_i = N^i $ by definition.
Thus, we know that $ d_i / d_{i - 1} = N $ obviously divides $ N $.
Since $ h_k = d_k / d_{k - 1} $, we know that $ h_k $ divides $ N $ for all $ k $.
\QED{}

\section{}
\label{sec:Question6}

Solving, we find that:
\begin{align}
    A &= \begin{bmatrix} 3 & 1 \\ -1 & 2 \end{bmatrix} \\
    P^{-1} &= \begin{bmatrix} 3 & 1 \\ -1 & 0 \end{bmatrix} \\
    Q^{-1} &= \begin{bmatrix} 1 & -2 \\ 0 & 1 \end{bmatrix} \\
    Q &= \begin{bmatrix} 1 & 2 \\ 0 & 1 \end{bmatrix} \\
    A' &= \begin{bmatrix} 1 & 0 \\ 0 & 7 \end{bmatrix}
\end{align}
to make $ A = P^{-1} A' Q^{-1} $ true.

which was easy to do, since we had already computed these values from Question~\ref{sec:Question1}.
\\ \\
Thus, our new commensurable basis for $ \mathbb{Z}^2 $ is $ \begin{bmatrix} 3 \\ -1 \end{bmatrix}, \begin{bmatrix} 1 \\ 0 \end{bmatrix} $, as these are the columns of $ P^{-1} $.
Computing, we also find that:
\[
P^{-1} A' = AQ = \begin{bmatrix} 3 & 7 \\ -1 & 0 \end{bmatrix}
\]
and thus the columns form the new basis for $ \Lambda $.
Formally, we say that the new basis for $ \Lambda $ is $ \mathbb{Z}^2 $ is $ \begin{bmatrix} 3 \\ -1 \end{bmatrix}, \begin{bmatrix} 7 \\ 0 \end{bmatrix} $
\QED{}

\section{}
\label{sec:Question7}

Denote $ G $ to be the group presented by the given matrix $ A $.
We yield the Smith Normal Form of this matrix to be $ 2 I_3 $.
Thus the invariants are $ h_1 = h_2 = h_3 = 2 $.
By Theorem 22.2 in Lecture 22 on Slide 7, we know that $ G = C_{h_1} \times C_{h_2} \times C_{h_3} = C_2 \times C_2 \times C_2 $.
\QED{}

\end{document}
